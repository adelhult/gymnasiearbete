Den färdiga prototypen som syns i figur \ref{fig:final_unit} fungerar nu som önskat. Efter att en användare yttrat frasen ”Hey Mycroft” går det att ställa frågor och sedan få frågorna besvarade av enheten. I jämförelse med bland annat Googles röstprodukter så är dock Mycroft långsam och svårförstådd. Men detta var förväntat, det finns inte heller någon enkel åtgärd då Mycroft-projektet är det bästa alternativet baserad på öppen källkod för tillfället. 

Ett relativt lågt brus låter dock konstant från högtalarna när enheten är ansluten till en strömkälla. Detta beror på att ljudet från enkortsdatorn överförs analogt direkt till förstärkaren, även brus från störningar förstärks alltså också. För att förbättra ljudet bör såldes ett digitalt ljudkort användas. Ett sådant överför ljudet digitalt och därav utan större störningar till förstärkaren. Detta prövades också men fungerade inte eftersom drivrutinerna till mikrofonkortet och ljudkortet (som i detta fall användes) inte var kompatibla. För att få ett bättre ljud helt utan brus rekommenderas det därför att använda sig av annan hårdvara. Ett mikrofonkort som också har inbyggt stöd för digitalt ljud är den enklaste lösningen. Ett annat alternativ är också förstås att hitta två komponenter som är kompatibla med varandra.

Angående designen av höljet kan också små förbättringar göras, bland annat så är det lite för trångt mellan högtalarna för att de ska passa perfekt. Om man önskar att bevara högtalarna i mitten av varje sida bör därför sidorna av enheten vara bredare. Så lite som fyra millimeter bredare sidor skulle troligtvis räcka för högtalarna inte ska stöta i varandra.

Sammanfattningsvis så har detta gymnasiearbete presenterat en fungerande metod för att designa och bygga en smart högtalare som är baserad på öppen källkod. Som ovan nämnt är dock denna prototyp ej helt fri från fel, materialet och tillvägagångs\-sättet kan på flera sätt förändras för att konstruera en bättre enhet.